\documentclass[letterpaper,sans,12pt]{moderncv}

\moderncvstyle{banking}

\nopagenumbers{}

\usepackage[scale=0.75]{geometry}

\name{Juan Pedro}{Villa Isaza}
\address{El Retiro, Colombia}
\homepage{www.jpvillaisaza.com}
\social[github]{jpvillaisaza}

\begin{document}

\makecvtitle

\section{Experience}

\cventry
  {Apr 2017--Present}
  {Haskell Developer}
  {Stack Builders}
  {Remote}
  {}
  {
    Responsibilities and achievements include:
    \begin{itemize}
      \item
        Contributed to client projects using Haskell (with frameworks
        such as Servant and Yesod), PureScript, and Reason, along with
        tools like Rclone.
      \item
        Led the internal Haskell training program, restructuring the
        course curriculum to align with Graham Hutton's
        \textit{Programming in Haskell}.
      \item
        Authored blog posts and tutorials, and assisted in content
        review for the company website.
    \end{itemize}
  }
\cventry
  {Sep 2014--Nov 2016}
  {Software Developer}
  {Stack Builders}
  {Quito, Ecuador}
  {}
  {
    Responsibilities and achievements include:
    \begin{itemize}
      \item
        Contributed to client projects using Haskell (with frameworks
        such as Servant, Snap, and Yesod).
      \item
        Supervised and managed small Haskell projects.
      \item
        Authored a series of blog posts and tutorials for the website.
      \item
        Initiated and led the tutorials initiative.
      \item
        Coordinated content creation efforts.
      \item
        Assisted in organizing and managing the Quito Lambda meetup.
    \end{itemize}
  }
\cventry
  {2009--2012}
  {Teaching Assistant}
  {Universidad EAFIT}
  {Medellín, Colombia}
  {}
  {
    \begin{itemize}
      \item
        Assisted in teaching CB0081 (Automata and Formal Languages).
    \end{itemize}
  }
\cventry
  {2010--2010}
  {Intern}
  {EAFIT Virtual, Universidad EAFIT}
  {Medellín, Colombia}
  {}
  {}

\section{Education}

\cventry
  {2006--2014}
  {Systems Engineer}
  {Universidad EAFIT}
  {Medellín, Colombia}
  {4.48/5}
  {
    \begin{itemize}
      \item
        Honorable mention for the undergraduate project \emph{Category
        Theory Applied to Functional Programming} (2014).
      \item
        First Class Honors (2006)
      \item
        Member of the Logic and Computation Research Group
        (2007--2014).
      \item
        Attended LERNET 2008, the International Summer School on
        Language Engineering and Rigorous Software Development in
        Piriápolis, Uruguay.
    \end{itemize}
  }
\cventry
  {2007--2011}
  {Mathematical Engineering}
  {Universidad EAFIT}
  {Medellín, Colombia}
  {Incomplete Degree}
  {
    \begin{itemize}
    \item
      Completed seven semesters of coursework.
    \end{itemize}
  }
\cventry
  {1993--2005}
  {High School Graduate}
  {Colegio San Ignacio}
  {Medellín, Colombia}
  {}
  {}

\section{Skills}

\cvitem{Programming Languages}
  {Haskell and functional programming languages}
\cvitem{Markup Languages}{HTML, \LaTeX, Markdown}
\cvitem{Databases}{PostgreSQL}
\cvitem{Tools}{Git, Jekyll}

\section{Projects}

\cventry
  {2016}
  {A license compatibility helper for Haskell}
  {Licensor}
  {Haskell}
  {}
  {
    \href{https://github.com/jpvillaisaza/licensor}
    {\texttt{https://github.com/jpvillaisaza/licensor}}
  }
\cventry
  {}
  {Colombian opinion articles aggregator}
  {Hermes}
  {Haskell}
  {}
  {
    \href{https://github.com/jpvillaisaza/hermes}
      {\texttt{github.com/jpvillaisaza/hermes}}
  }
\cventry
  {}
  {Fuel economy calculator}
  {Zenith}
  {Elm}
  {}
  {
    \href{https://github.com/jpvillaisaza/zenith}
      {\texttt{github.com/jpvillaisaza/zenith}}
  }

\section{Publications}

\cventry{2014}
  {Category Theory Applied to Functional Programming}
  {Systems Engineering undergraduate project}
  {}
  {}
  {
    \begin{itemize}
      \item
        Supervisor: Andrés Sicard Ramírez
      \item
        Website:
          \href
            {https://github.com/jpvillaisaza/cain}
            {\texttt{github.com/jpvillaisaza/cain}}
    \end{itemize}
  }

\subsection{Blog posts and tutorials}

\begin{itemize}
  \item
    \href
      {https://www.stackbuilders.com/blog/getting-started-with-haskell-projects-using-scotty/}
      {Getting started with Haskell projects using Scotty (2021)}
  \item
    \href
      {https://www.stackbuilders.com/blog/nonsense-getting-started-with-reason-and-reason-react/}
      {Nonsense! Getting started with Reason and React (2020)}
  \item
    \href
      {https://www.stackbuilders.com/blog/csv-encoding-decoding/}
      {CSV encoding and decoding in Haskell with Cassava (2016)}
  \item
    \href
      {https://www.stackbuilders.com/blog/reverse-reverse-theorem-proving-with-idris/}
      {Reverse, reverse: Theorem proving with Idris (2015)}
  \item
    \href
      {https://www.stackbuilders.com/blog/a-quickcheck-tutorial-generators/}
      {A QuickCheck tutorial: Generators (2015, updated 2024)}
  \item
    \href
      {https://www.stackbuilders.com/blog/the-hang-of-elm-hangman-and-functional-programming-with-elm/}
      {The hang of Elm: Hangman and functional programming with Elm (2015)}
  \item
    \href
      {https://www.stackbuilders.com/blog/errors-and-exceptions-in-haskell/}
      {Errors and exceptions in Haskell (2015)}
  \item
    \href
      {https://www.stackbuilders.com/blog/dr-hakyll-create-a-github-page-with-hakyll-and-circleci/}
      {Dr. Hakyll: Create a GitHub page with Hakyll and CircleCI (2015)}
  \item
    \href
      {https://www.stackbuilders.com/blog/obverse-versus-reverse-benchmarking-in-haskell-with-criterion/}
      {Obverse versus reverse: Benchmarking in Haskell with Criterion (2015)}
  % \item
  %   \href
  %     {https://www.stackbuilders.com/blog/how-to-program-it/}
  %     {How to program it (2015)}
  \item
    \href
      {https://www.stackbuilders.com/blog/hangman-imperative-functional-programming/}
      {Hangman: Imperative functional programming (2015)}
  \item
    \href
      {https://www.stackbuilders.com/blog/pattern-matching-wots-uh-the-deal/}
      {Pattern matching: Wot's... Uh the Deal? (2015)}
  % \item
  %   \href
  %     {https://www.stackbuilders.com/blog/the-notes-of-ghc/}
  %     {The notes of GHC (2015)}
  % \item
  %   \href
  %     {https://www.stackbuilders.com/blog/obverse-and-reverse/}
  %     {Obverse versus reverse (2015)}
  \item
    \href
      {https://www.stackbuilders.com/blog/the-weak-and-the-strong-functors/}
      {The weak and the strong: functors (2014)}
\end{itemize}

\subsection{Presentations}

% \cvitem{2013/09/25}{
%   \emph{Enoch: Just Enoch},
%   Logic and Computation Seminar,
%   Universidad EAFIT.
% }

% \cvitem{2013/04/19}{
%   \emph{Enoch: Not the father of Methuselah},
%   Logic and Computation Seminar,
%   Universidad EAFIT.
% }

% \cvitem{2013/03/22}{
%   \emph{Enoch: The father of Irad},
%   Logic and Computation Seminar,
%   Universidad EAFIT.
% }

% \cvitem{2013/02/22}{
%   \emph{Enoch: A son of Cain},
%   Logic and Computation Seminar,
%   Universidad EAFIT.
% }

\cvitem{2009}{
  \emph{Proofs $=$ Programs: Functional programming with dependent types},
  Días de la Ciencia Aplicada 2009,
  Universidad EAFIT.
}

\cvitem{}{
  \emph{Non-dependent Types for the Implementation of a Dependently
    Typed Functional Programming Language},
  Logic and Computation Seminar,
  Universidad EAFIT.
}

\cvitem{2007}{
  \emph{Regular Expressions Using Haskell},
  Logic and Computation Seminar,
  Universidad EAFIT.
}

\section{Interests}

\cvitem{}{Bird watching, cooking, reading, video games, writing}

\section{Languages}

\cvdoubleitem{Spanish}{Native}{French}{Beginner}
\cvdoubleitem{English}{Fluent}{}{}

% \cventry{2014}{Moralities of Everyday Life}{Coursera}{}{91.7/100}{}
% \cventry{2010--2011}{Jóvenes pioneros}{Periódico El Colombiano}{Medellín}{}{}

\vspace*{\fill}
\begin{footnotesize}
  \cvitem{Version}{\texttt{\version} (\today)}
\end{footnotesize}

\end{document}
